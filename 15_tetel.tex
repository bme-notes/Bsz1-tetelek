\documentclass[]{article}
\usepackage{lmodern}
\usepackage{amssymb}
\usepackage{amsmath}
\usepackage{polyglossia}
\usepackage{listings}
\usepackage{tcolorbox}
\usepackage{etoolbox}
\usepackage{setspace}
\usepackage{framed}
\usepackage[a4paper,margin=2cm,footskip=.5cm]{geometry}
\newcommand{\R}{\mathbb{R}}
\newcommand{\Rn}[1]{$\mathbb{R}^{#1}$}
\newcommand{\Und}[1]{\underline{#1}}
\definecolor{shadecolor}{gray}{0.9}
%opening 
\title{Bevezetés a Számításelméletbe 1.\\{\large 15. tétel}}
\author{Hegyi Zsolt}
\begin{document}
\maketitle{}
\begin{shaded}
EULER-FÉLE $\varphi$-FÜGGVÉNY Definíció: Ha $n \geq 2$ egész, akkor az $1,\ldots,n-1$ számok között n-hez relatív prímek számát $\varphi(n)$-el jelöljük. Az $n\mapsto\varphi(n)$ függvényt Euler-féle $\varphi$ függvénynek nevezzük.
\end{shaded}
\begin{framed}
EULER-FÉLE $\varphi$-FÜGGVÉNYRE KÉPLET Tétel: Legyen az $n > 1$ egész kanonikus alakja $n = p_1^{\alpha_1} \cdot \ldots \cdot p_k^{\alpha_k}$ Ekkor $$\varphi(n) = \left(p_1^{\alpha_1} - p_1^{\alpha_1-1}\right) \cdot \left(p_2^{\alpha_2} - p_2^{\alpha_2-1}\right) \cdot \ldots \cdot
\left(p_k^{\alpha_k} - p_k^{\alpha_k-1}\right)$$
\end{framed}
\begin{leftbar}
Biz: 130-131. oldal Szeszlér-jegyzet.
\end{leftbar}
\begin{shaded}
REDUKÁLT MARADÉKRENDSZER Definíció: Az $R = \{c_1, \ldots, c_k\}$ számhalmaz redukált maradékrendszer modulo m, ha a következő feltételeknek eleget tesz:
\begin{itemize}
\item ($c_i, m$) = 1 minden i = 1, $\ldots$, k esetén;
\item $c_i \not \equiv c_j$ (mod m) bármely $i \neq j, 1 \leq i, j \leq k$ esetén;
\item $k = \varphi(m).$
\end{itemize}
\end{shaded}
\begin{framed}
REDUKÁLT MARADÉKRENDSZER Tétel: Legyen $R = \{c_1, \ldots, c_k\}$ redukált maradékrendszer modulo m és legyen tetszőleges egész, melyre (a,m) = 1. Ekkor az $R' = \{a\cdot c_1, \ldots, a \cdot c_k\}$ halmaz szintén redukált maradékrendszer modulo m.
\end{framed}
\begin{leftbar}
Biz: 132. oldal Szeszlér-jegyzet.
\end{leftbar}
\begin{framed}
EULER-FERMAT Tétel: Ha az a és m $\geq$ 2 egészekre (a,m) = 1, akkor $a^{\varphi(m) \equiv 1}$ (mod m).
\end{framed}
\begin{leftbar}
Biz: 132-133. oldal Szeszlér-jegyzet.
\end{leftbar}
\begin{framed}
"KIS" FERMAT-Tétel: Ha p prím és a tetszőleges egész, akkor $a^p \equiv a$ (mod p).
\end{framed}
\begin{leftbar}
Biz: 133. oldal Szeszlér-jegyzet.
\end{leftbar}
A tételhez hozzátartozik diofantikus illetve két kongruenciából álló kongruenciarendszerek megoldása is, konkrét példán.
\end{document}
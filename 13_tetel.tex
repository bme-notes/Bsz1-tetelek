\documentclass[]{article}
\usepackage{lmodern}
\usepackage{amssymb}
\usepackage{amsmath}
\usepackage{polyglossia}
\usepackage{listings}
\usepackage{tcolorbox}
\usepackage{etoolbox}
\usepackage{setspace}
\usepackage{framed}
\usepackage[a4paper,margin=2cm,footskip=.5cm]{geometry}
\newcommand{\R}{\mathbb{R}}
\newcommand{\Rn}[1]{$\mathbb{R}^{#1}$}
\newcommand{\Und}[1]{\underline{#1}}
\definecolor{shadecolor}{gray}{0.9}
%opening 
\title{Bevezetés a Számításelméletbe 1.\\{\large 13. tétel}}
\author{Hegyi Zsolt}
\begin{document}
\maketitle{}
\begin{shaded}
OSZTHATÓSÁG Definíció: Azt mondjuk, hogy az $a \in \mathbb{Z}$ egész osztója $b \in \mathbb{Z}$ egésznek, ha létezik olyan $c \in \mathbb{Z}$, melyre $a \cdot c = b$. Ugyanezt fejezzük ki, ha b-t az a többszörösének mondjuk. \\
Jelölés: $a|b$, ha pedig a nem osztója b-nek, $a\!\not|\ b$.\\Az a valódi osztója b-nek, ha a|b fennál és $1 < |a| < |b|$.
\end{shaded}
\begin{shaded}
PRÍMSZÁM Definíció: A $p \in \mathbb{Z}$ egészt prímszámnak nevezzük, ha $|p| > 1$ és p-nek nincsen valódi osztója. Tehát $p = a \cdot b$ csak akkor lehetséges, ha $a = \pm 1$ vagy $b = \pm 1$. Ha $|p| > 1$ és p nem prím, akkor összetett számnak nevezzük.
\end{shaded}
\begin{framed}
SZÁMELMÉLET ALAPTÉTELE Tétel: Minden 1-től, 0-tól és (-1)-től különböző egész szám felbontható prímek szorzatára és ez a felbontás a tényezők sorrendjétől és előjelétől eltekintve egyértelmű.
\end{framed}
\begin{leftbar}
Biz: 114. oldal Szeszlér-jegyzet.
\end{leftbar}
\begin{framed}
PRÍMEK SZÁMOSSÁGA Tétel: A prímek száma végtelen.
\end{framed}
\begin{leftbar}
Biz: 117. oldal Szeszlér-jegyzet.
\end{leftbar}
\begin{framed}
SZOMSZÉDOS PRÍMEK KÖZTI HÉZAGOK Tétel: Minden $N > 1$ egészhez találhatóak olyan $p < q$ prímek, hogy p és q között nincs további prím és $q-p>N$.
\end{framed}
\begin{leftbar}
Biz: 117-118. oldal Szeszlér-jegyzet.
\end{leftbar}
\begin{framed}
NAGY PRÍMSZÁMTÉTEL Tétel: $\pi(n) \approx \frac{n}{\ln{n}}$ vagyis $\lim_{n\to\infty} \frac{\pi(n)}{\frac{n}{\ln{n}}} = 1$
\end{framed}
\begin{shaded}
KONGRUENCIA Definíció: legyenek $a,b,m\in\mathbb{Z}$ tetszőleges egészek. Azt mondjuk, hogy a konguens b-vel modulo m, ha a-t és b-t m-mel maradékosan osztva azonos maradékokat kapunk. Az m számot a kongruencia modulusának nevezzük. \\
Jelölés: $a \equiv b$ (mod m)

\end{shaded}
\begin{framed}
Tetszőleges  $a,b,m\in\mathbb{Z}$ egészekre $a \equiv b$ (mod m) akkor és csak akkor igaz, ha $m|a-b$.
\end{framed}
\begin{leftbar}
Biz: 119. oldal Szeszlér-jegyzet.
\end{leftbar}
\begin{framed}
ALAPMŰVELETEK KONGRUENCIÁKKAL Tétel: T.f.h. $a \equiv b$ (mod m) és $c \equiv d$ (mod m) fennállnak a,b,c,d,m egészekre és $k \geq 1$ tetszőleges. Ekkor igazak az alábbiak:
\begin{itemize}
\item $a + c \equiv b + d$ (mod m)
\item $a - c \equiv b - d$ (mod m)
\item $a \cdot c \equiv b \cdot d$ (mod m)
\item $a^k \equiv b^k$ (mod m)
\end{itemize}
\end{framed}
\begin{leftbar}
Biz: 120. oldal Szeszlér-jegyzet.
\end{leftbar}
\begin{framed}
KONGRUENCIA Tétel: Legyenek a,b,c,m tetszőlegesek és $d = (c,m)$ (lnko). Ekkor $a\cdot c \equiv b\cdot d$ (mod m) akkor és csak akkor igaz, ha $a \equiv b$ (mod $\frac{m}{d})$.
\end{framed}
\begin{leftbar}
Biz: 120-121. oldal Szeszlér-jegyzet.
\end{leftbar}
\end{document}

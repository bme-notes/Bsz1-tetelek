\section{10. tétel}

\begin{tetel}{LINEÁRIS TRANSZFORMÁCIÓ INVERTÁLHATÓSÁGA}
Egy $f$: \Rn{n} $\rightarrow$ \Rn{n} lineáris transzformáció akkor és csak akkor invertálható, ha $det[f] \neq 0$. Ha pedig ez a feltétel fennáll, akkor $[f^{-1}] = [f]^{-1}$ - vagyis az $f^{-1}$ inverz transzformáció mátrixa az $f$ mátrixának az inverze.
\end{tetel}

\begin{bizonyitas}{}
100. oldal Szeszlér-jegyzet.
\end{bizonyitas}

\begin{definicio}{MAGTÉR, KÉPTÉR}
Legyen $f$: \Rn{n} $\rightarrow$ \Rn{k} lineáris leképezés. $f$ magterének nevezzük és Ker$f$-fel jelöljük azon \Rn{n}-beli vektorok halmazát, melyeknek a képe az \Rn{k}-beli nullvektor:
$$Kerf = \{\Und{x}\in \R^n : f(\Und{x}) = \Und{0}\}$$
$f$ képterének nevezzük és Im$f$-fel jelöljük azon \Rn{k}-beli vektorok halmazát, melyek megkaphatók (legalább) egy alkalmas \Rn{n}-beli vektor $f$-fel vett képeként.
$$Imf = \{\Und{y} \in \R^k : \exists\Und{x} \in \R^n, f(\Und{x}) = \Und{y}\}$$
\end{definicio}

\begin{tetel}{MAGTÉR, KÉPTÉR ALTÉR VOLTA}
Legyen $f$: \Rn{n} $\rightarrow$ \Rn{k} lineáris leképezés. Ekkor:
\begin{itemize}
\item Ker$f\leq\R^{n}$, vagyis Ker$f$ altér \Rn{n}-ben;
\item Im$f\leq\R^k$, vagyis Im$f$ altér \Rn{k}-ban.
\end{itemize}
\end{tetel}

\begin{bizonyitas}{}
96. oldal Szeszlér-jegyzet.
\end{bizonyitas}

\begin{tetel}{DIMENZIÓTÉTEL}
Ha $f$: \Rn{n} $\rightarrow$ \Rn{k} lineáris leképezés, akkor dim Ker$f$ + dim Im$f$ = n.
\end{tetel}

\begin{bizonyitas}{}
97. oldal Szeszlér-jegyzet.
\end{bizonyitas}


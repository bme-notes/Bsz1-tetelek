\section{11. tétel}

\begin{tetel}{BÁZISTRANSZFORMÁCIÓ}
Legyen $f$: \Rn{n} $\rightarrow$ \Rn{n} lineáris transzformáció és B egy ($n \times n$)-es mátrix, melynek az oszlopai bázist alkotnak \Rn{n}-ben. Jelölje $g$: \Rn{n} $\rightarrow$ \Rn{n} azt a függvényt, mely minden $\Und{x} \in \R^n$ esetén $[\Und{x}]_B$-hez $[f(\Und{x})]_B$-t rendeli. Ekkor $g$ is lineáris transzformáció, melynek a mátrixa $[g] = B^{-1} \cdot [f] \cdot B$.
\end{tetel}

\begin{bizonyitas}{}
102. oldal Szeszlér-jegyzet.
\end{bizonyitas}

\begin{tetel}{BÁZISTRANSZFORMÁCIÓ}
Legyen $h$: \Rn{n} $\rightarrow$ \Rn{n} az a függvény, mely minden $\Und{x} \in \R^n$ esetén $[\Und{x}]_B$-hez $\Und{x}$-et rendeli. Ekkor h lineáris transzformáció, amelynek mátrixa $[h] = B$.
\end{tetel}

\begin{bizonyitas}{}
102. oldal Szeszlér-jegyzet.
\end{bizonyitas}

\begin{definicio}{LINEÁRIS TRANSZF. ADOTT BÁZIS SZERINT}
Legyen $f$: \Rn{n} $\rightarrow$ \Rn{n} lineáris transzformáció és B bázis \Rn{n}-ben. Ekkor a $g$: $[\Und{x}]_B \mapsto [f(\Und{x})]_B$ lineáris transzformáció mátrixát az $f$ transzformáció B bázis szerinti mátrixának nevezzük.\\
Jelölés: $[f]_B$
\end{definicio}

\begin{tetel}{BÁZISTRANSZFORMÁCIÓ KISZÁMÍTÁSA}
Az $[f]_B$ mátrix i-edik oszlopa egyenlő az $[f(\Und{b}_i)]_B$ koordinátavektorral minden $1 \leq i \leq n$ esetén, ahol $f$: \Rn{n} $\rightarrow$ \Rn{n} tetszőleges lineáris transzformáció és $B = \{\Und{b}_1, \ldots, \Und{b}_n\}$ bázis \Rn{n}-ben.
\end{tetel}

\begin{bizonyitas}{}
104. oldal Szeszlér-jegyzet.
\end{bizonyitas}


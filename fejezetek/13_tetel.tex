\section{13. tétel}

\begin{definicio}{OSZTHATÓSÁG}
Azt mondjuk, hogy az $a \in \mathbb{Z}$ egész osztója $b \in \mathbb{Z}$ egésznek, ha létezik olyan $c \in \mathbb{Z}$, melyre $a \cdot c = b$. Ugyanezt fejezzük ki, ha b-t az a többszörösének mondjuk. \\
Jelölés: $a|b$, ha pedig a nem osztója b-nek, $a\!\not|\ b$.\\Az a valódi osztója b-nek, ha a|b fennál és $1 < |a| < |b|$.
\end{definicio}

\begin{definicio}{PRÍMSZÁM}
A $p \in \mathbb{Z}$ egészt prímszámnak nevezzük, ha $|p| > 1$ és p-nek nincsen valódi osztója. Tehát $p = a \cdot b$ csak akkor lehetséges, ha $a = \pm 1$ vagy $b = \pm 1$. Ha $|p| > 1$ és p nem prím, akkor összetett számnak nevezzük.
\end{definicio}

\begin{tetel}{SZÁMELMÉLET ALAPTÉTELE}
Minden 1-től, 0-tól és (-1)-től különböző egész szám felbontható prímek szorzatára és ez a felbontás a tényezők sorrendjétől és előjelétől eltekintve egyértelmű.
\end{tetel}

\begin{bizonyitas}{}
114. oldal Szeszlér-jegyzet.
\end{bizonyitas}

\begin{tetel}{PRÍMEK SZÁMOSSÁGA}
A prímek száma végtelen.
\end{tetel}

\begin{bizonyitas}{}
117. oldal Szeszlér-jegyzet.
\end{bizonyitas}

\begin{tetel}{SZOMSZÉDOS PRÍMEK KÖZTI HÉZAGOK}
Minden $N > 1$ egészhez találhatóak olyan $p < q$ prímek, hogy p és q között nincs további prím és $q-p>N$.
\end{tetel}

\begin{bizonyitas}{}
117-118. oldal Szeszlér-jegyzet.
\end{bizonyitas}

\begin{tetel}{NAGY PRÍMSZÁMTÉTEL}
$\pi(n) \approx \frac{n}{\ln{n}}$ vagyis $\lim_{n\to\infty} \frac{\pi(n)}{\frac{n}{\ln{n}}} = 1$
\end{tetel}

\begin{definicio}{KONGRUENCIA}
legyenek $a,b,m\in\mathbb{Z}$ tetszőleges egészek. Azt mondjuk, hogy a konguens b-vel modulo m, ha a-t és b-t m-mel maradékosan osztva azonos maradékokat kapunk. Az m számot a kongruencia modulusának nevezzük. \\
Jelölés: $a \equiv b$ (mod m)

\end{definicio}

\begin{tetel}{}
Tetszőleges $a,b,m\in\mathbb{Z}$ egészekre $a \equiv b$ (mod m) akkor és csak akkor igaz, ha $m|a-b$.
\end{tetel}

\begin{bizonyitas}{}
119. oldal Szeszlér-jegyzet.
\end{bizonyitas}

\begin{tetel}{ALAPMŰVELETEK KONGRUENCIÁKKAL}
T.f.h. $a \equiv b$ (mod m) és $c \equiv d$ (mod m) fennállnak a,b,c,d,m egészekre és $k \geq 1$ tetszőleges. Ekkor igazak az alábbiak:
\begin{itemize}
\item $a + c \equiv b + d$ (mod m)
\item $a - c \equiv b - d$ (mod m)
\item $a \cdot c \equiv b \cdot d$ (mod m)
\item $a^k \equiv b^k$ (mod m)
\end{itemize}
\end{tetel}

\begin{bizonyitas}{}
120. oldal Szeszlér-jegyzet.
\end{bizonyitas}

\begin{tetel}{KONGRUENCIA}
Legyenek a,b,c,m tetszőlegesek és $d = (c,m)$ (lnko). Ekkor $a\cdot c \equiv b\cdot c$ (mod m) akkor és csak akkor igaz, ha $a \equiv b$ (mod $\frac{m}{d})$.
\end{tetel}

\begin{bizonyitas}{}
120-121. oldal Szeszlér-jegyzet.
\end{bizonyitas}


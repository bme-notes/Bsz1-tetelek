\section{9. tétel}

\begin{shaded}
LINEÁRIS LEKÉPEZÉS Definíció: Az $f$: \Rn{n} $\rightarrow$ \Rn{k} függvényt lineáris leképezésnek hívjuk, ha létezik egy olyan ($k \times n$)-es mátrix, melyre $f(\Und{x}) = A \cdot \Und{x}$ teljesül minden $\Und{x} \in$ \Rn{n} esetén. Az n = k esetben f-et lineáris transzformációnak is nevezzük. Ha $f\::\:$\Rn{n} $\rightarrow$ \Rn{k} lineáris leképezés és $f(\Und{x}) = A \cdot \Und{x}$ minden $\Und{x} \in$ \Rn{n}-re, akkor azt mondjuk, hogy a mátrixa A.\\
Jelölés: $A = [f]$.
\end{shaded}
\begin{framed}
LINEÁRIS LEKÉPEZÉS FELTÉTELE Tétel: Az $f$: \Rn{n} $\rightarrow$ \Rn{k} függvény akkor és csak akkor lineáris leképezés, ha:
\begin{itemize}
\item $f(\Und{x} + \Und{y}) = f(\Und{x}) + f(\Und{y})$ igaz minden $\Und{x},\Und{y} \in$ \Rn{n} esetén;
\item $f(\lambda \cdot \Und{x}) = \lambda \cdot f(\Und{x})$ igaz minden $\Und{x} \in$ \Rn{n} és $\lambda \in \R$ esetén.
\end{itemize}
Ha pedig f teljesíti ezt a két tulajdonságot, akkor az $[f]$ egyértelmű és azonos azzal a ($k \times n$)-es mátrixszal, melynek minden $1 \leq i \leq n$ esetén az i-edik oszlopa $f(\Und{e}_i)$.
\end{framed}
\begin{leftbar}
Biz: 92. oldal Szeszlér-jegyzet.
\end{leftbar}
\begin{framed}
LINEÁRIS LEKÉPEZÉSEK SZORZATA Tétel: Legyenek $f$: \Rn{n} $\rightarrow$ \Rn{k} és $g$: \Rn{k} $\rightarrow$ \Rn{m} lineáris leképezések. Ekkor ezeknek a $g \circ f$ szorzata is lineáris leképezés, melyre $[g \circ f] = [g] \cdot [f]$.
\end{framed}
\begin{leftbar}
Biz: 94. oldal Szeszlér-jegyzet.
\end{leftbar}
\begin{framed}
ADDÍCIÓS TÉTELEK Tétel: Tetszőleges $\alpha$ és $\beta$ szögekre teljesülnek az alábbi összefüggések:
$$\sin(\alpha + \beta) = \sin\alpha\cdot\cos\beta + \cos\alpha\cdot\sin\beta$$
$$\cos(\alpha + \beta) = \cos\alpha\cdot\cos\beta - \sin\alpha\cdot\sin\beta$$
\end{framed}
\begin{leftbar}
Biz: 95. oldal Szeszlér-jegyzet.
\end{leftbar}

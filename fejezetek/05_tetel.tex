\section{5. tétel}

\begin{definicio}{
DETERMINÁNS Definíció}: Legyen adott egy ($n \times n$)-es A mátrix. Az A minden bástyaelhelyezésére
szorozzuk össze az azt alkotó n elemet, majd a szorzatot lássuk el előjellel
a következő szabály szerint: ha a bástyaelhelyezésnek megfelelő permutáció inverziószáma
páros, akkor az előjel legyen pozitív, ha viszont páratlan az inverziószám,
akkor az előjel legyen negatív. Az így kapott n! darab, n tényezős előjelezett szorzat
összegét az A determinánsának nevezzük. \\
Jelölés: |A| vagy detA.
\end{definicio}
\begin{tetel}{
DETERMINÁNS ALAPTULAJDONSÁGAI Tétel}: Legyen A egy ($n \times n$)-es mátrix,\\
Ha A-nak van csupa 0 elemet tartalmazó sora vagy oszlopa, akkor detA $= 0$.\\
Ha A felsőháromszög-mátrix vagy alsóháromszög-mátrix, akkor a determinánsa a főátlóbeli elemek szorzata: $$det A = a_{1,1}\cdot a_{2,2}\cdot \ldots\cdot a_{n,n}$$
\end{tetel}
\begin{leftbar}
Biz: Az első állítás bizonyítása azonnal következik a determináns definíciójából: mivel mind az n! db. szorzat tartalmaz elemet abból a sorból/oszlopból, amelyiknek minden tagja 0, ezért minden szorzat értéke és ezek összege is 0 lesz.\\
A második állítás bizonyításához vegyük A felsőháromszög-mátrixot. A bástyaelhelyezések akkor nem tartalmaznak 0 elemet, ha az első oszlopból az első elemet, a második oszlopból a második elemet, választjuk ki (a többit nem válaszhatnánk ki) és így tovább... Az így kapott permutáció inverziószáma 0, így pozitív előjelű ez a tag, és mivel ez az egyetlen tag, amiben nem szerepel 0, ezért ez lesz az előjeles összeg eredménye. Ezt megismételve az oszlop és a sor szavak megcserélésével megkapjuk ugyanezt a bizonyítást az alsóháromszög-mátrixra is.
\end{leftbar}
\begin{tetel}{
DETERMINÁNS ALAPTULAJDONSÁGAI Tétel}: Legyen A ($n\times n$)-es mátrix, $\lambda \in \R$ skalár, $1 \leq i,j\leq n, i \neq j$ egészek.\\
Ha A egy sorát/oszlopát megszorozzuk $\lambda$-val, akkor a kapott A' mátrix determinánsa $\lambda$-szorosa A-énak:$$detA' = \lambda\cdot detA$$
Ha A két sorát/oszlopát felcseréljük, akkor a kapott A' mátrix determinánsa ellentetje az A-énak:
$$detA' = (-1)\cdot detA$$
Ha A i-edik sorát helyettesítjük sajátmagának és a j-edik sor $\lambda$-szorosának összegével, akkor a kapott A' mátrix determinánsa megegyezik A-éval:
$$detA' = detA$$ Ugyanez igaz oszlopokra is.
\end{tetel}
\begin{leftbar}
Lásd 48-50. oldal Szeszlér-jegyzet.
\end{leftbar}
\begin{tetel}{
DETERMINÁNS KISZÁMOLÁSA -  GAUSS ELIMINÁCIÓVAL\\
Bemenet - $n\times n$ mátrix}.\\
\textbf{0. lépés}: i <- 1, D <- 1\\
\textbf{1. lépés}:
\begin{itemize}
\item Ha $a_{i,j} = 0$, akkor folytassuk \textbf{2. lépésnél}.
\item Szorozzuk meg i-edik sort $\frac{1}{a_{i,j}}$-vel.
\item D <- $D \cdot a_{i,j}$
\item Ha $i = n$, akkor PRINT "detA =", D; STOP.
\item Minden $i < t \leq n$ esetén adjuk a t-edik sorhoz az i-edik sor ($-a_{t,i}$)-szeresét.
\item i <- $i+1$
\item Folytassuk az \textbf{1. lépésnél}.
\end{itemize}
\textbf{2. lépés}
\begin{itemize}
\item Ha $i < n$ és van olyan $i < t \leq k$, melyre $a_{t,i} \neq 0$, akkor:
\begin{itemize}
\item Cseréljük fel az i-edik sort a t-edik sorral.
\item D <- $(-1)\cdot D$
\item Folytassuk az \textbf{1. lépésnél}.
\end{itemize}
\item PRINT "detA = 0"; STOP.
\end{itemize}
\end{tetel}
\begin{tetel}{
TRANSZPONÁLT DETERMINÁNSA Tétel}: Minden A négyzetes mátrixra $detA^T = detA$
\end{tetel}
\begin{leftbar}
A bizonyítás megtalálható a Szeszlér-jegyzet 65-66. oldalán.
\end{leftbar}

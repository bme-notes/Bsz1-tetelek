\section{6. tétel}

\begin{tetel}{KIFEJTÉSI Tétel}
Ha az ($n \times n$)-es A mátrix valamelyik sorának, vagy oszlopának minden elemét megszorozzuk a hozzá tartozó előjeles aldetermináns értékével és a kapott n darab kéttényezős szorzatot összeadjuk, akkor az A determinánsának értékét kapjuk.
\end{tetel}

\begin{bizonyitas}{}
A Szeszlér-jegyzet 55-58. oldalán.
\end{bizonyitas}

\begin{definicio}{MÁTRIX}
Adott $k,n\geq1$ egészek esetén ($k\times n$)-es mátrixnak nevezünk egy k sorból és n oszlopból álló táblázatot, melynek minden cellájában egy valós szám áll. A ($k\times n$)-es mátrixok halmazát \Rn{k\times n} jelöli. Az A mátrix i-edik sorának és j-edik oszlopának kereszteződésében álló elemet $a_{i,j}$ jelöli. Az \Rn{k\times n}-en értelmezett, ``+''-al jelölt összeadást és tetszőleges $\lambda \in \R$ esetén ``$\cdot$''-tal jelölt skalárral való szorzást tudjuk értelmezni. Nem, nem fogom leírni, hogyan néz ki egy szorzás/összeadás $k \times n$-es mátrixon.
\end{definicio}

\begin{tetel}{MÁTRIXMŰVELETEK}
Legyen A,B,C $\in$ \Rn{k\times n} és $\lambda, \mu \in \R$. Ekkor igazak az alábbiak:\\
A mátrixösszeadás asszociatív és kommutatív.\\
A mátrixszorzás asszociatív és disztributív. (NEM KOMMUTATÍV)!
\end{tetel}

\begin{definicio}{TRANSZPONÁLT}
A ($k\times n$)-es A mátrix transzponáltjának nevezzük az ($n\times k$)-as B mátrixot, ha $b_{i,j} = a_{j,i}$ teljesül minden $1 \leq i \leq n$ és $1 \leq j \leq k$ esetén.\\
Jelölés: $B = A^T$
\end{definicio}

\begin{definicio}{MÁTRIXSZORZÁS}
A ($k\times n$)-es A ($n\times m$)-es B mátrixok szorzatának nevezzük és $A\cdot B$-vel jelöljük azt a ($k\times m$)-es C mátrixot, melyre minden $1 \leq i \leq k$ és $1 \leq j \leq m$ esetén
$$c_{i,j} = a_{i,1}\cdot b_{1, j}+\ldots+a_{i,n}\cdot b_{n,j}$$
\end{definicio}

Ha az A és B mátrixokra $A \cdot B$ szorzat létezik, akkor $B^T \cdot A^T$ is létezik és $(A\cdot B)^T = B^T \cdot A^T$.

\begin{tetel}{DETERMINÁNSOK SZORZÁSTÉTELE}
Bármely A és B ($n\times n$)-es mátrixokra: $$det(A\cdot B)=detA \cdot detB$$
\end{tetel}


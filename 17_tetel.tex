\documentclass[]{article}
\usepackage{lmodern}
\usepackage{amssymb}
\usepackage{amsmath}
\usepackage{polyglossia}
\usepackage{listings}
\usepackage{tcolorbox}
\usepackage{etoolbox}
\usepackage{setspace}
\usepackage{framed}
\usepackage[a4paper,margin=2cm,footskip=.5cm]{geometry}
\newcommand{\R}{\mathbb{R}}
\newcommand{\Rn}[1]{$\mathbb{R}^{#1}$}
\newcommand{\Und}[1]{\underline{#1}}
\definecolor{shadecolor}{gray}{0.9}
%opening 
\title{Bevezetés a Számításelméletbe 1.\\{\large 17. tétel}}
\author{Hegyi Zsolt}
\begin{document}
\maketitle{}
\begin{shaded}
NEM KELL - SEGÉDDEFINÍCIÓK: Legyenek A és B tetszőleges halmazok és $f$ : A $\to$ B egy függvény. Az $f$ függvényt
\begin{itemize}
\item injektívnek nevezzük, ha bármely $x_1, x_2 \in A, x_1 \neq x_2$ esetén $f(x_1) \neq f(x_2)$.
\item szürjektívnek nevezzük, ha bármely $y \in B$ esetén létezik olyan $x \in A$, melyre $f(x) = y$.
\item bijektívnek nevezzük, ha injektív és szürjektív.
\end{itemize}
\end{shaded}
\begin{shaded}
HALMAZOK SZÁMOSSÁGÁNAK EGYENLŐSÉGE Definíció: Azt mondjuk, hogy a (tetszőleges) A és B halmazok számossága egyenlő, ha létezik egy $f$ : A $\to$ B bijekció.\\
Jelölés: |A| = |B|
\end{shaded}
\begin{shaded}
$\mathbb{N}$ SZÁMOSSÁGA: Az A halmazt megszámlálhatóan végtelennek nevezzük, ha a számossága egyenlő a természetes számok halmazáéval (tehát |A| = |$\mathbb{N}$|).\\
Jelölés: |A| = $\aleph_0$
\end{shaded}
\begin{framed}
$\mathbb{Q}$, $\mathbb{Z}$ SZÁMOSSÁGA Tétel: Az egész számok $\mathbb{Z}$ halmaza és a racionális számok $\mathbb{Q}$ halmaza egyaránt megszámlálhatóan végtelen.
\end{framed}
\begin{leftbar}
Biz: 161. oldal Szeszlér-jegyzet
\end{leftbar}
\begin{framed}
CANTOR Tétel: A valós számok $\mathbb{R}$ halmaza nem megszámlálhatóan végtelen, vagyis $$|\mathbb{N}| \neq |\mathbb{R}|$$
\end{framed}
\begin{leftbar}
Biz: 162-164. oldal Szeszlér-jegyzet.
\end{leftbar}
\begin{shaded}
$\mathbb{R}$ SZÁMOSSÁGA Definíció: Az A halmazt kontinuum számosságúnak nevezzük, ha a számossága egyenlő a valós számok halmazáéval (vagyis |A| = |$\mathbb{R}$|).\\
Jelölés: |A| = c.
\end{shaded}
A (0,1) nyílt intervallum is kontinuum számosságú. Ld. 163. oldal Szeszlér-jegyzet.
\begin{shaded}
Legyenek A és B (tetszőleges) halmazok.
\begin{itemize}
\item Azt mondjuk, hogy A számossága kisebb vagy egyenlő B számoságánál, ha létezik $f\::\:A \to B$ injektív függvény.\\
Jelölés: $|A| \leq |B|$
\item Azt mondjuk, hogy A számossága kisebb B számosságánál, ha $|A| \leq |B|$, de $|A| \neq |B|$.\\
Jelölés: $|A| < |B|$
\end{itemize}
\end{shaded}
\begin{framed}
CANTOR-BERNSTEIN Tétel: Az A és B halmazokra |A| = |B| akkor és csak akkor igaz, ha $|A| \leq |B|$ és $|B| \leq |A|$
\end{framed}
\begin{framed}
$\mathbb{Q}$ SZÁMOSSÁGA Tétel: $|\mathbb{Q}| = |\mathbb{N}|$
\end{framed}
\begin{leftbar}
Biz: 167. oldal Szeszlér-jegyzet.
\end{leftbar}
\begin{shaded}
HATVÁNYHALMAZ Definíció: Tetszőleges A halmaz hatványhalmazának nevezzük az A összes részhalmaza által alkotott halmazt.\\
Jelölés: P(A)
\end{shaded}
\begin{framed}
(ismét?) CANTOR-Tétel: Minden A halmazra |A| < |P(A)|.
\end{framed}
\begin{leftbar}
Biz: 169. oldal Szeszlér-jegyzet.
\end{leftbar}
\begin{framed}
$\mathbb{N}$ HATVÁNYHALMAZÁNAK SZÁMOSSÁGA Tétel: |P($\mathbb{N}$)| = |$\mathbb{R}$|
\end{framed}
\begin{leftbar}
Biz: 170-171. oldal Szeszlér-jegyzet.
\end{leftbar}
\end{document}
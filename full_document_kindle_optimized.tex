\documentclass[ebook]{memoir}
\usepackage{lmodern}
\usepackage{amssymb}
\usepackage{amsmath}
\usepackage{polyglossia}
\usepackage{listings}
\usepackage{tcolorbox}
\usepackage{etoolbox}
\usepackage{setspace}
\usepackage{framed}
\usepackage[hidelinks]{hyperref}
\usepackage{breqn}
\usepackage{microtype}
\usepackage[a6paper,margin=1pt]{geometry}
\newcommand{\R}{\mathbb{R}}
\newcommand{\Rn}[1]{$\mathbb{R}^{#1}$}
\newcommand{\Und}[1]{\underline{#1}}
\definecolor{shadecolor}{HTML}{eeeeee} %Kindle-optimized color
\setcounter{secnumdepth}{0} %this automagically removes extra numbering toc and sections
\renewcommand{\contentsname}{Tartalomjegyzék}
\renewcommand{\tocheadstart}{}
\renewcommand{\printtoctitle}[1]{\Large\textbf{#1}}
\renewcommand{\aftertoctitle}{\vspace{10pt}}
\renewcommand{\cftdot}{} 
\title{Bevezetés a számításelméletbe 1. tételsor}
\author{Zsolt Hegyi\\}
\date{\textbf{Kellemes vizsgázást!}} %hekk
\begin{document}
\maketitle
\tableofcontents*
\newpage
\section{1. tétel}
\input{01_tetel_stripped}
\newpage
\section{2. tétel}
\input{02_tetel_stripped}
\newpage
\section{3. tétel}
\input{03_tetel_stripped}
\newpage
\section{4. tétel}
\input{04_tetel_stripped}
\newpage
\section{5. tétel}
\input{05_tetel_stripped}
\newpage
\section{6. tétel}
\input{06_tetel_stripped}
\newpage
\section{7. tétel}
\input{07_tetel_stripped}
\newpage
\section{8. tétel}
\input{08_tetel_stripped}
\newpage
\section{9. tétel}
\input{09_tetel_stripped}
\newpage
\section{10. tétel}
\input{10_tetel_stripped}
\newpage
\section{11. tétel}
\input{11_tetel_stripped}
\newpage
\section{12. tétel}
\input{12_tetel_stripped}
\newpage
\section{13. tétel}
\input{13_tetel_stripped}
\newpage
\section{14. tétel}
\input{14_tetel_stripped}
\newpage
\section{15. tétel}
\input{15_tetel_stripped}
\newpage
\section{16. tétel}
\input{16_tetel_stripped}
\newpage
\section{17. tétel}
\input{17_tetel_stripped}
\end{document}
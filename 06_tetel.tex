\documentclass[]{article}
\usepackage{lmodern}
\usepackage{amssymb}
\usepackage{amsmath}
\usepackage{polyglossia}
\usepackage{listings}
\usepackage{tcolorbox}
\usepackage{etoolbox}
\usepackage{setspace}
\usepackage{framed}
\usepackage[a4paper,margin=2cm,footskip=.5cm]{geometry}
\newcommand{\R}{\mathbb{R}}
\newcommand{\Rn}[1]{$\mathbb{R}^{#1}$}
\newcommand{\Und}[1]{\underline{#1}}
\definecolor{shadecolor}{gray}{0.9}
%opening 
\title{Bevezetés a Számításelméletbe 1.\\{\large 6. tétel}}
\author{Hegyi Zsolt}
\begin{document}
\maketitle{}
\begin{framed}
KIFEJTÉSI Tétel: Ha az ($n \times n$)-es A mátrix valamelyik sorának, vagy oszlopának minden elemét megszorozzuk a hozzá tartozó előjeles aldetermináns értékével és a kapott n darab kéttényezős szorzatot összeadjuk, akkor az A determinánsának értékét kapjuk.
\end{framed}
\begin{leftbar}
Biz: A Szeszlér-jegyzet 55-58. oldalán.
\end{leftbar}
\begin{shaded}
MÁTRIX Definíció: Adott $k,n\geq1$ egészek esetén ($k\times n$)-es mátrixnak nevezünk egy k sorból és n oszlopból álló táblázatot, melynek minden cellájában egy valós szám áll. A ($k\times n$)-es mátrixok halmazát \Rn{k\times n} jelöli. Az A mátrix i-edik sorának és j-edik oszlopának kereszteződésében álló elemet $a_{i,j}$ jelöli. Az \Rn{k\times n}-en értelmezett, "+"-al jelölt összeadást és tetszőleges $\lambda \in \R$ esetén "$\cdot$"-tal jelölt skalárral való szorzást tudjuk értelmezni. Nem, nem fogom leírni, hogyan néz ki egy szorzás/összeadás $k \times n$-es mátrixon.
\end{shaded}
\begin{framed}
MÁTRIXMŰVELETEK Tétel: Legyen A,B,C $\in$ \Rn{k\times n} és $\lambda, \mu \in \R$. Ekkor igazak az alábbiak:\\
A mátrixösszeadás asszociatív és kommutatív.\\
A mátrixszorzás asszociatív és disztributív. (NEM KOMMUTATÍV)!
\end{framed}
\begin{shaded}
TRANSZPONÁLT Definíció: A ($k\times n$)-es A mátrix transzponáltjának nevezzük az ($n\times k$)-as B mátrixot, ha $b_{i,j} = a_{j_i}$ teljesül minden $1 \leq i \leq n$ és $1 \leq j \leq k$ esetén.\\
Jelölés: $B = A^T$
\end{shaded}
\begin{shaded}
MÁTRIXSZORZÁS Definíció: A ($k\times n$)-es A ($n\times m$)-es B mátrixok szorzatának nevezzük és $A\cdot B$-vel jelöljól azt a ($k\times m$)-es C mátrixot, melyre minden $1 \leq i \leq k$ és $1 \leq j \leq m$ esetén
$$c_{i,j} = a_{i,1}\cdot b_{1, j}+\ldots+a_{i,n}\cdot b_{n,j}$$
\end{shaded}
Ha az A és B mátrixokra $A \cdot B$ szorzat létezik, akkor $B^T \cdot A^T$ is létezik és $(A\cdot B)^T = B^T \cdot A^T$.
\begin{framed}
DETERMINÁNSOK SZORZÁSTÉTELE Tétel: Bármely A és B ($n\times n$)-es mátrixokra: $$det(A\cdot B)=detA \cdot detB$$
\end{framed}
\end{document}
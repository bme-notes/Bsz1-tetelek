\documentclass[]{article}
\usepackage{lmodern}
\usepackage{amssymb}
\usepackage{amsmath}
\usepackage{polyglossia}
\usepackage{listings}
\usepackage{tcolorbox}
\usepackage{etoolbox}
\usepackage{setspace}
\usepackage{framed}
\usepackage[a4paper,margin=2cm,footskip=.5cm]{geometry}
\newcommand{\R}{\mathbb{R}}
\newcommand{\Rn}[1]{$\mathbb{R}^{#1}$}
\newcommand{\Und}[1]{\underline{#1}}
\definecolor{shadecolor}{gray}{0.9}
%opening 
\title{Bevezetés a Számításelméletbe 1.\\{\large 10. tétel}}
\author{Hegyi Zsolt}
\begin{document}
\maketitle{}
\begin{framed}
LINEÁRIS TRANSZFORMÁCIÓ INVERTÁLHATÓSÁGA Tétel: Egy $f$: \Rn{n} $\rightarrow$ \Rn{n} lineáris transzformáció akkor és csak akkor invertálható, ha $det[f] \neq 0$. Ha pedig ez a feltétel fennál, akkor $[f^{-1}] = [f]^{-1}$ - vagyis az $f^{-1}$ inverz transzformáció mátrixa az $f$ mátrixának az inverze.
\end{framed}
\begin{leftbar}
Biz: 100. oldal Szeszlér-jegyzet.
\end{leftbar}
\begin{shaded}
MAGTÉR, KÉPTÉR Definíció: Legyen $f$: \Rn{n} $\rightarrow$ \Rn{k} lineáris leképezés. $f$ magterének nevezzük és Ker$f$-fel jelöljük azon \Rn{n}-beli vektorok halmazát, melyeknek a képe az \Rn{k}-beli nullvektor:
$$Kerf = \{\Und{x}\in \R^n : f(\Und{x}) = \Und{0}\}$$
$f$ képterének nevezzük és Im$f$-fel jelöljük azon \Rn{k}-beli vektorok halmazát, melyek megkaphatók (legalább) egy alkalmas \Rn{n}-beli vektor $f$-fel vett képeként.
$$Imf = \{\Und{y} \in \R^k : \exists\Und{x} \in \R^n, f(\Und{x}) = \Und{y}\}$$
\end{shaded}
\begin{framed}
MAGTÉR, KÉPTÉR ALTÉR VOLTA Tétel: Legyen $f$: \Rn{n} $\rightarrow$ \Rn{k} lineáris leképezés. Ekkor:
\begin{itemize}
\item Ker$f\leq\R^{n}$, vagyis Ker$f$ altér \Rn{n}-ben;
\item Im$f\leq\R^k$, vagyis Im$f$ altér \Rn{k}-ban.
\end{itemize}
\end{framed}
\begin{leftbar}
Biz: 96. oldal Szeszlér-jegyzet.
\end{leftbar}
\begin{framed}
DIMENZIÓTÉTEL Tétel: Ha $f$: \Rn{n} $\rightarrow$ \Rn{k} lineáris leképezés, akkor dim Ker$f$ + dim Im$f$ = n.
\end{framed}
\begin{leftbar}
Biz: 97. oldal Szeszlér-jegyzet.
\end{leftbar}
\end{document}
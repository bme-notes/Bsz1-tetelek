\documentclass[]{article}
\usepackage{lmodern}
\usepackage{amssymb}
\usepackage{amsmath}
\usepackage{polyglossia}
\usepackage{listings}
\usepackage{tcolorbox}
\usepackage{etoolbox}
\usepackage{setspace}
\usepackage{framed}
\usepackage[a4paper,margin=2cm,footskip=.5cm]{geometry}
\newcommand{\R}{\mathbb{R}}
\newcommand{\Rn}[1]{$\mathbb{R}^{#1}$}
\newcommand{\Und}[1]{\underline{#1}}
\definecolor{shadecolor}{gray}{0.9}
%opening 
\title{Bevezetés a Számításelméletbe 1.\\{\large 16. tétel}}
\author{Hegyi Zsolt}
\begin{document}
\maketitle{}
\begin{shaded}
POLINOMIÁLIS FUTÁSIDEJŰ ALGORITMUS: Definíció: (vázlatos) A algoritmust polinomiális futásidejűnek tekintjük, ha n méretű bemenethez tartozó $f$(n) függvényre, mely az algoritmus lépésszámát határozza meg, a következő MINDEN n esetén fennáll:
$$f(n) \leq c\cdot n^k$$ ahol c és k rögzített konstansok.
\end{shaded}
A Szeszlér-jegyzet 137-141. oldalán található egy hosszas mese a számelméleti algoritmusokról, ezek közül az ALAPMŰVELETEK és a HATVÁNYOZÁS, valamint HATVÁNYOZÁS MODULO M a fontosak.
\begin{framed}
PRÍMTESZTELÉS Fermat-teszt:
\begin{description}
\item[Bemenet]m egész
\item[0. lépés]k $\leftarrow$ 1
\item[1. lépés]Generáljunk véletlen számot 1 és m-1 közt.
\item[2. lépés]Euklideszi-algoritmussal számoljuk ki (a,m) értékét. Ha (a,m)$\neq1$, m NEM prím, STOP.
\item[3. lépés]Számítsuk ki $a^{m-1}$ (mod m) értékét ISMÉTELT NÉGYZETRE EMELÉSEK MÓDSZERÉVEL. Ha nem 1, m NEM prím, STOP.
\item[4. lépés]Ha k = 100, m VALÓSZÍNŰLEG prím.
\item[5. lépés]k $\leftarrow$ k+1, vissza \textbf{1. lépés}hez
\end{description}
\end{framed}
\begin{framed}
FERMAT-TESZT ÁRULÓK SZÁMA Tétel: Ha m > 1 összetett szám és m-nek van árulója, akkor az 1 és m közötti, m-hez relatív prím számoknak legalább a fele áruló.
\end{framed}
\begin{leftbar}
Biz: 147. oldal Szeszlér-jegyzet.
\end{leftbar}
\begin{shaded}
Az m > 1 összetett számot univerzális álprímnek, vagy más néven Carmichael-számnak nevezzük, ha nincsen árulója, vagyis, ha minden 1 < a < m, (a,m) = 1 esetén $a^{m-1} \equiv 1$ (mod m).
\end{shaded}
Nyilvános kulcsú titkosítás és az RSA-algoritmus:\\
152-154. oldal Szeszlér-jegyzet.
\end{document}

\documentclass[]{article}
\usepackage{lmodern}
\usepackage{amssymb}
\usepackage{amsmath}
\usepackage{polyglossia}
\usepackage{listings}
\usepackage{tcolorbox}
\usepackage{etoolbox}
\usepackage{setspace}
\usepackage{framed}
\usepackage[a4paper,margin=2cm,footskip=.5cm]{geometry}
\newcommand{\R}{\mathbb{R}}
\newcommand{\Rn}[1]{$\mathbb{R}^{#1}$}
\newcommand{\Und}[1]{\underline{#1}}
\definecolor{shadecolor}{gray}{0.9}
%opening 
\title{Bevezetés a Számításelméletbe 1.\\{\large 12. tétel}}
\author{Hegyi Zsolt}
\begin{document}
\maketitle{}
\begin{shaded}
DIAGONÁLIS MÁTRIX Definíció: Az A ($n \times n$)-es mátrix akkor nevezzük diagonális mátrixnak, ha minden $i \neq j$ esetén $a_{i,j} = 0$ teljesül.
\end{shaded}
\begin{framed}
KAPCSOLAT SAJÁTÉRTÉK ÉS LINEÁRIS LEKÉPEZÉSEK KÖZT Valami:
Legyen $B = \{\Und{b}_1, \ldots, \Und{b}_n\}$ tetszőleges bázis és t.f.h. az $[f]_B$ mátrix diagonális, a főátlóban álló elemeket jelölje sorban $\lambda_1, \ldots, \lambda_n$. Ekkor az $[f]_B$ i-edik oszlopa $\lambda_i\cdot\Und{e}_i$-vel egyenlő, ebből kifolyólag $[f(\Und{b}_i)]_B = \lambda_i \cdot \Und{e}_i$. Ez viszont azt jelenti, hogy $f(b_i) = 0\cdot\Und{b}_1 + \ldots + \lambda_i \cdot \Und{b}_i + \ldots + 0 \cdot\Und{b}_n$, vagyis $f(b_i) = \lambda_i \cdot \Und{b}_i$.\\
ÖSSZEFOGLALVA: $[f]_B$ akkor lesz diagonális, ha B minden tagjára $f(b_i) = \lambda_i \cdot \Und{b}_i$ teljesül valamilyen $\lambda$ skalárral.
\end{framed}
\begin{shaded}
SAJÁTÉRTÉK, SAJÁTVEKTOR Definíció: Legyen A egy ($n \times n$)-es mátrix.
\begin{itemize}
\item A sajátértékének nevezzük a $\lambda \in \R$ skalárt, ha létezik olyan $\Und{x} \in \R^n,\: \Und{x} \neq \Und{0}$ vektor, melyre $A \cdot \Und{x} = \lambda\cdot \Und{x}$
\item A sajátvektorának nevezzük az $\Und{x} \in \R^n$ vektort, ha $\Und{x} \neq \Und{0}$ és létezik olyan $\lambda \in \R$ skalár, melyre $A \cdot \Und{x} = \lambda\cdot \Und{x}$
\end{itemize}
Rövidítve: Ha $A\cdot\Und{x} = \lambda\cdot\Und{x}$ teljesül és $\Und{x} \neq \Und{0}$, akkor $\lambda$ a sajátértéke és $\Und{x}$ a sajátvektora az A-nak.
\end{shaded}
\begin{framed}
SAJÁTÉRTÉK MEGHATÁROZÁSA Tétel: A négyzetes A mátrixnak a $\lambda \in \R$ skalár akkor és csak akkor sajátértéke, ha $det(A - \lambda\cdot E) = 0$.
\end{framed}
\begin{leftbar}
Biz: 106. oldal Szeszlér-jegyzet.
\end{leftbar}
\begin{shaded}
KARAKTERISZTIKUS POLINOM Definíció: Az ($n \times n$)-es A mátrix karakterisztikus polinomjának nevezzük a $det(A - \lambda\cdot E)$ determináns értékét, ahol $\lambda$ változó.\\
Jelölés: $k_A(\lambda)$.
\end{shaded}
A sajátérték definíciója átfogalmazva az előző tétel és definíció felhasználásával: A mátrix sajátértékei a $k_A(\lambda)$ karakterisztikus polinom gyökei, tehát a $k_A(\lambda) = 0$ egyenlet megoldásai. Az algebra egyik tétele szerint tehát n-edfokú polinomnak legfeljebb n gyöke lehet, amiből következik, hogy ($n \times n$)-es mátrixnak legfeljebb n sajátértéke van.
\end{document}

\documentclass[]{article}
\usepackage{lmodern}
\usepackage{amssymb}
\usepackage{amsmath}
\usepackage{polyglossia}
\usepackage{listings}
\usepackage{tcolorbox}
\usepackage{etoolbox}
\usepackage{setspace}
\usepackage{framed}
\usepackage[hidelinks]{hyperref}
\usepackage[a4paper,margin=2cm,footskip=.5cm]{geometry}
\newcommand{\R}{\mathbb{R}}
\newcommand{\Rn}[1]{$\mathbb{R}^{#1}$}
\newcommand{\Und}[1]{\underline{#1}}
\definecolor{shadecolor}{HTML}{eeeeee} %Kindle-optimized color
\setcounter{secnumdepth}{0} %this automagically removes extra numbering toc and sections
\renewcommand{\contentsname}{Tartalomjegyzék}
\title{Bevezetés a számításelméletbe 1. tételsor}
\author{Zsolt Hegyi}
\date{}
\begin{document}
\maketitle
\noindent A tételsor Szeszlér Dávid fantasztikus előadásai és jegyzete alapján készült.
\\
\\
\noindent Álljon itt egy névsor azokról, akik a megjegyzéseikkel, támogatásukkal és munkájukkal érdemben részt vettek e tételsor létrehozásában illetve javításában:
\\
\indent Bálint Ádám, Bereczki Márk, Bognár Márton, Bokros Bálint, Braun Márton, Dános Péter, Hanusch Róbert, az IRC-s baráti társaságom, Koczka Tamás, Kormány Zsolt, a KSZK reszort tagjai, Müller András, Nagy "Sid" Jenő, Rostás Balázs.
\\
\\
\noindent \textbf{Kellemes vizsgázást!} %és ne felejtsétek otthon a síkosítót...
\tableofcontents{}
\newpage
\section{1. tétel}
\input{01_tetel_stripped}
\newpage
\section{2. tétel}
\input{02_tetel_stripped}
\newpage
\section{3. tétel}
\input{03_tetel_stripped}
\newpage
\section{4. tétel}
\input{04_tetel_stripped}
\newpage
\section{5. tétel}
\input{05_tetel_stripped}
\newpage
\section{6. tétel}
\input{06_tetel_stripped}
\newpage
\section{7. tétel}
\input{07_tetel_stripped}
\newpage
\section{8. tétel}
\input{08_tetel_stripped}
\newpage
\section{9. tétel}
\input{09_tetel_stripped}
\newpage
\section{10. tétel}
\input{10_tetel_stripped}
\newpage
\section{11. tétel}
\input{11_tetel_stripped}
\newpage
\section{12. tétel}
\input{12_tetel_stripped}
\newpage
\section{13. tétel}
\input{13_tetel_stripped}
\newpage
\section{14. tétel}
\input{14_tetel_stripped}
\newpage
\section{15. tétel}
\input{15_tetel_stripped}
\newpage
\section{16. tétel}
\input{16_tetel_stripped}
\newpage
\section{17. tétel}
\input{17_tetel_stripped}
\end{document}
